% Author: Akshay Khadse
% Roll No.: 153079011
% Email: akshaykhadse@iitb.ac.in
% Date: 01-10-2016

\documentclass[12pt, a4paper]{article}

\usepackage{amsmath}
\usepackage{graphicx}
\usepackage{float}
\usepackage{url}
\usepackage[bookmarks=true]{hyperref}
\usepackage{bookmark}

\begin{document}

\title{Van der Pol Oscillator}
\author{Akshay Khadse\\
	Roll No. 153079011\\
	\texttt{akshaykhadse@iitb.ac.in}
	}
\date{\today}
\maketitle

\tableofcontents
\listoffigures

\input{getparams.txt}

\begin{abstract}
Van der Pol oscillator is a non conservative oscillator\cite{wiki} with non linear damping. It evolves according to following second order differential equation:
\begin{equation} \label{sec-ord-eq}
\frac{d^2x}{dt^2}-\mu(1-x^2)\frac{dx}{dt}+x=0
\end{equation}
where $x$ is position which is function of time $t$ and $\mu$, a scalsr parameter indicating nonlinearity and strength of damping.
\end{abstract}

\section{Source Code}
\subsection{Repository}
The source code for Van der Pol Oscillator can be found at following repository:\\
\url{https://github.com/akshaykhadse/vanderpol.git}

\subsection{Dependencies}
The souce code at above mentioned repo is based on:
\begin{itemize}
\item Python 3.5.1
\item Jupyter 4.2.0
\item pdfTeX 3.14159265-2.6-1.40.16
\item matplotlib 1.5.3
\item numpy 1.11.2
\item scipy 0.18.1
\item GNU Make 4.1
\end{itemize}

\subsection{Changing Parameters}
Following are the parameters used in current solution of Van der Pol Equations\\
$\mu = \getmu{}$\\
$X0 = \getinitial{}$

To change these values, edit the \texttt{\# Parameters} section of \texttt{van\_der\_pol.py} file.

\section{Background}
The Van der Pol oscillator was originally proposed by the Dutch electrical engineer and physicist Balthasar van der Pol while he was working at Philips.\cite{wiki:1} Van der Pol found stable oscillations, which he subsequently called relaxation-oscillations\cite{wiki:3} and are now known as a type of limit cycle in electrical circuits employing vacuum tubes. When these circuits were driven near the limit cycle, they become entrained, i.e. the driving signal pulls the current along with it. Van der Pol and his colleague, van der Mark, reported in the September 1927 issue of Nature that at certain drive frequencies an irregular noise was deterministic chaos.\cite{wiki:5}
The Van der Pol equation has a long history of being used in both the physical and biological sciences. For instance, in biology, Fitzhugh and Nagumo extended the equation in a planar field as a model for action potentials of neurons. The equation has also been utilised in seismology to model the two plates in a geological fault, and in studies of phonation to model the right and left vocal fold oscillators.

\section{Two Dimensional Form}
Lienard's theorem can be used to prove that the system has a limit cycle. Applying the Liénard transformation $y = x- x^3/3 - \dot{x}/\mu$ , where the dot indicates the time derivative, the Van der Pol oscillator can be written in its two-dimensional form:
\begin{equation} \label{2dformx}
\dot{x} = \mu\left(x - \frac{1}{3}x^3 - y \right)
\end{equation}
\begin{equation} \label{2dformy}
\dot{y} = \frac{1}{\mu}x
\end{equation}

Another commonly used form based on the transformation $ y = \dot{x} $ leads to:
\begin{equation} \label{2dform2xnew}
\dot{x} = y
\end{equation}
\begin{equation} \label{2dform2ynew}
\dot{y} = \mu(1-x^2)y - x
\end{equation}

\section{Results for Unforced Oscillator}
Two interesting regimes for the characteristics of the unforced oscillator are:

When $\mu = 0$, i.e. there is no damping function, the equation becomes:
\begin{equation} \label{simple}
\frac{d^2x}{dt^2} + x = 0
\end{equation}

This is a form of the simple harmonic oscillator, and there is always conservation of energy.
When $\mu > 0$, the system will enter a limit cycle. Near the origin $x = dx/dt = 0$, the system is unstable, and far from the origin, the system is damped.

Following is the result with parameters $\mu = \getmu{}$ and $X0 = \getinitial{}$ while solving the Unforced equation problem.
\begin{figure}[H]
\centering
\includegraphics[width=0.8\textwidth]{vanderpol-1.png}
\caption{States vs Time (Roll No. 153079011)}
\label{fig:states}
\end{figure}

The figure \ref{fig:states} shows the solution plot of Van der Pol Equations. The plot depicts $x$ vs. $t$ and $y$ vs. $y$ respectively.

\begin{figure}[H]
\centering
\includegraphics[width=0.8\textwidth]{vanderpol-2.png}
\caption{Phase Portrait (Roll No. 153079011)}
\label{fig:phase}
\end{figure}

The figure \ref{fig:phase} shows the phase plot for the solution in figure \ref{fig:states}. The plot represents relation ship between $x$ and $y$.

\section{Animation}
Follow the link to view animation.
\href{./153079011.html}{Click to View Animation}

\bibliographystyle{plain}
\bibliography{source/references}
\end{document}
